% Guassian integrals

\begin{problem}[\label{PureQuadraticGaussInt}Purely quadratic case]
\begin{equation}
I_{\text{pure gauss.}}(a) = \int_{-\infty}^{\infty} {\rm{d}}x\, e^{ax^2}
\end{equation} 
A way to solve this integral is to intorduce $I(a)^2$, go to a polar coordinate representation by a simple substitution, then integrate over the polar angle $\phi$ and finally take the square root of the result.
\begin{align}
	I(a)^2 &= \int_{-\infty}^{\infty} {\rm{d}}x_1 \int_{-\infty}^{\infty} {\rm{d}} x_2 e^{ax_1^2} e^{ax_2^2} \\
	&= \iint_{-\infty}^{\infty} {\rm{d}}x_1 {\rm{d}}x_2 \, e^{a(x_1^2 + x_2^2)}
\end{align}
Now one can go from ${\rm{d}}x_1 {\rm{d}}x_2$ to ${\rm{d}}r {\rm{d}}\phi$, by the $x_1 = r \cos(\phi)$ and $x_2 = r\sin(\phi)$. Meaning that $x_1^2 + x_2^2 = r^2$ replaces the exponent. Calculating the Jacobian for the change of the metric
\begin{equation}
	J = \left| 
	\begin{matrix}
	\frac{\partial x_1}{\partial r} & \frac{\partial x_1}{\partial \phi} \\
	\frac{\partial x_2}{\partial r} & \frac{\partial x_2}{\partial \phi} 
	\end{matrix}
	\right| = \left|
	\begin{matrix}
	\cos(\phi) & -r\sin(\phi)\\
	\sin(\phi) & r\cos(\phi)\\
	\end{matrix}
	\right| = r
\end{equation}

\begin{align}
	I(a)^2 &= \int_{0}^{\infty} {\rm{d}}r \int_{0}^{2\pi}{\rm{d}}\phi \, re^{a(r^2)} \\
	&= 2\pi \int_{0}^{\infty} {\rm{d}}r \, re^{a(r^2)}
\end{align}
It is convinient to make another substitution $q = -r^2 \rightarrow {\rm{d}}q = -2r {\rm{d}}r$.
\begin{align}
	I(a)^2 &= -2\pi \int_{0}^{\infty} {\rm{d}}q \, \frac{1}{2} e^{-a\,q} \\
	&= \frac{\pi}{-a} \\
	\Aboxed{\int_{-\infty}^{\infty} {\rm{d}}x\, e^{ax^2} &= \sqrt{\frac{\pi}{-a}}}\\
	 &\text{ assuming that $\Real{a} \leq 0$} \nonumber
\end{align}
\end{problem}

\begin{problem}[\label{GeneralQuadraticGaussInt}General Gaussian Integral]
\begin{equation}
	I_{\text{Gen.gauss.}}(a, b, c) = \int_{-\infty}^{\infty} {\rm{d}}x\, e^{ax^2 + bx + c}
\end{equation}
The second order polinomial in the exponent can be written as a square: $ \alpha(x + \beta)^2 - \gamma = ax^2 + bx + c$. Identifying the coefficients on the l.h.s. we find that $\alpha = a;\, \beta = \frac{b}{2a};$ and $\gamma = c - \frac{b^2}{4a^2}$. Now rewrite the integral using the squared form of the polynomial
\begin{equation}
	I(a, b, c) = \int {\rm{d}}x\, e^{a(x + \frac{b}{2a})^2 + c - \frac{b^2}{4a^2}}
\end{equation}
where the last two terms in the exponent are just constant factors which can be placed outside of the integral, leaving the $a(x + \frac{b}{2a})^2$ term. Substitute in $y = x + \frac{b}{2a}$ (${\rm{d}}y = {\rm{d}}x$)
\begin{equation}
	I(a, b, c) = e^{c - \frac{b^2}{4a^2}} \int_{-\infty}^{\infty} {\rm{d}}y\, e^{ay^2}.
\end{equation}
Integral on the l.h.s. is the same as in problem \ref{PureQuadraticGaussInt}, giving the answer as
\begin{equation}
	\boxed{\int_{-\infty}^{\infty} {\rm{d}}x\, e^{ax^2 + bx + c} = e^{c - \frac{b^2}{4a^2}} \sqrt{\frac{\pi}{-a}}}
\end{equation}
\end{problem}






