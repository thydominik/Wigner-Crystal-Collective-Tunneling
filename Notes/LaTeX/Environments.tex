% Latex Environments:
	% Update the original 'Environments.tex' file then copy it to the project that you work on, to keep an updated and uniform latex sintax, ty!

% Notes elements:
	% Theorem
		\newtheorem{theorem}{Theorem}[section]		%\begin{} and \end{} 'theorem', with printed 'Theorem ###' where the numbering will depend on the section it is placed in
	% Corollary
		\newtheorem{corollary}{Corollary}[theorem]
	% Definition
		\newtheorem{definition}{Definition}[section]
	% Lemma
		\newtheorem{lemma}[theorem]{Lemma}
	% Remark
		\theoremstyle{remark}
		\newtheorem*{remark}{Remark}
		
	\theoremstyle{definition}
		\newtheorem{problem}{Problem}[section]
		\newtheorem{example}{Example}[section] %\begin{} and \end{} 'example', with printed 'Example ###' where the numbering will depend on the section it is placed in
		\newtheorem{exercise}{Exercise}[section] 	% Initially for Bevezető kalkulus notes
		\newtheorem{recexercise}{recommended Exercise}[section]		% Initially for Bevezető kalkulus notes
	% NOTE BOX:
	\usetikzlibrary{arrows,calc,shadows.blur}
		\tcbuselibrary{skins}
		\newtcolorbox{note}[1][]{%
			enhanced jigsaw,
			colback=gray!10!white,%
			colframe=gray!80!black,
			size=small,
			boxrule=1pt,
			title=\textbf{Note:},
			halign title=flush center,
			coltitle=black,
			breakable,
			drop shadow=black!00!white,
			attach boxed title to top left={xshift=1cm,yshift=-\tcboxedtitleheight/2,yshifttext=-\tcboxedtitleheight/2},
			minipage boxed title=1.5cm,
			boxed title style={%
					colback=white,
					size=fbox,
					boxrule=1pt,
					boxsep=2pt,
					underlay={%
							\coordinate (dotA) at ($(interior.west) + (-0.5pt,0)$);
							\coordinate (dotB) at ($(interior.east) + (0.5pt,0)$);
							\begin{scope}
								\clip (interior.north west) rectangle ([xshift=3ex]interior.east);
								\filldraw [white, blur shadow={shadow opacity=00, shadow yshift=-.75ex}, rounded corners=2pt] (interior.north west) 			rectangle (interior.south east);
							\end{scope}
%							\begin{scope}[gray!80!black]
%								\fill (dotA) circle (2pt);
%								\fill (dotB) circle (2pt);
%							\end{scope}
						},
				},
			#1,
		}
		\newcommand{\nt}[1]{\begin{note}#1\end{note}}