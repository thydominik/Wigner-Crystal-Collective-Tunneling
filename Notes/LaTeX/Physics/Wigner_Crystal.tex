Wigner crystals are a unique type of solid that is formed when a dense two-dimensional electron gas is confined to a small area at low temperatures. These crystals were first proposed by physicist Eugene Wigner in 1934 as a way to describe the behavior of electrons in a metal under certain conditions. In a Wigner crystal, the electrons are arranged in a periodic lattice structure, much like atoms in a conventional crystal.\\

The formation of a Wigner crystal occurs when the density of electrons in a two-dimensional system becomes very high, typically on the order of $10^{11}$ electrons per square centimeter. At these densities, the electrons begin to interact strongly with each other, and their repulsive Coulomb interactions become dominant. This can cause the electrons to form a stable lattice structure, with each electron occupying a specific site in the lattice.\\

One of the most intriguing features of Wigner crystals is that they exhibit a variety of unusual physical properties that are not seen in ordinary solids. For example, because the electrons in a Wigner crystal are strongly correlated, they can behave as a collective entity rather than as individual particles. This can lead to a variety of interesting phenomena, including the appearance of exotic magnetic states and the emergence of unusual transport properties.\\

Wigner crystals have been studied extensively in recent years using a variety of experimental techniques, including scanning tunneling microscopy, transport measurements, and optical spectroscopy. These studies have provided important insights into the behavior of strongly correlated systems and have helped to shed light on a wide range of fundamental physical phenomena.\\

The original article on Wigner crystals was published by Eugene Wigner in 1934, and it provided a theoretical framework for understanding the behavior of electrons in a metal at low temperatures. Wigner's calculations were based on the assumption that the electrons in the metal were free to move around, but that they also interacted with each other through their Coulombic repulsion.\\

To analyze the behavior of these electrons, Wigner used a technique called the "method of paired electrons," which involves pairing up the electrons in the system and considering their joint probability distribution. This distribution describes the likelihood of finding two electrons at specific locations and with specific momenta, and it is a key quantity in the theory of Wigner crystals.\\

Wigner's calculations showed that as the density of electrons in the metal increased, the joint probability distribution of the paired electrons began to exhibit a periodic oscillatory pattern, indicating the formation of a crystalline lattice structure. This lattice structure was due to the repulsive interactions between the electrons, which caused them to arrange themselves in a regular pattern.\\

Wigner also calculated the energy of the Wigner crystal, which is related to the stability of the lattice structure. His calculations showed that the energy of the crystal increased with the number of electrons in the system, but it increased at a slower rate than the energy of an ideal gas of non-interacting electrons. This suggested that the Wigner crystal was a stable, low-energy state of the system, in contrast to the highly excited, high-energy states of an ideal gas.\\

Overall, Wigner's calculations provided a foundational framework for understanding the behavior of strongly correlated electron systems and the emergence of Wigner crystals. His work has since been expanded upon by many other researchers, and Wigner crystals continue to be an active area of research in condensed matter physics today.

