% Documentclass and packages
\documentclass[11pt,a4paper]{article}
\usepackage[english]{babel} % This package manages culturally-determined typographical (and other) rules

\usepackage{hyperref} % Makes references, links, toc clickable with changable color
\hypersetup{
    colorlinks,
    citecolor=black,
    filecolor=black,
    linkcolor=black,
    urlcolor=black
}

\usepackage{amsmath}	% Lot of math input
\numberwithin{equation}{subsection}	% Makes the numbering of the equations numbered by the section/appendix (###) -> (#.##)

\usepackage{import} % to include pdf not as a newpage document

\usepackage{bibunits} % makes it possible to tie bibliographies to certain sections and other units

% Title, Author, Date
\title{Theoretical modeling of the collective tunneling of a Wigner necklace\\}
\author{Dominik szombathy}
\date{\today}


\begin{document}
\maketitle

\begin{abstract}

\end{abstract}

%\pagestyle{empty}
\tableofcontents

%\setcounter{page}{1}
\pagestyle{plain}

\section{Wigner Crystals}

\section{Path Integral Formalism}


	\subsection{Free Particle}
	\subsection{Harmonic Oscillator}
	\subsection{One particle tunneling}
		\subsubsection{One Particle Hamiltonian}
		\subsubsection{Action Integral Calculation}
		\subsubsection{Tunneling calculation in quartic potentials}

\section{Experimental Considerations}
	\subsection{Modeling the Potential and Fitting}
	\subsection{Polarization data}
	\subsection{Scaling}

\section{Many Body Tunneling Calculations - Instanton}
	\subsection{Quartic Hamiltonian model}
		\subsubsection{Rescaling and Dimensionless Hamiltonian}	
	\subsection{Milnikov method}
	\subsection{Classical Equilibrium Positions}
	\subsection{Harmonic Oscillator Eigenfrequencies adn Eigenmodes}
	\subsection{Trajectory Calculation}
	\subsection{Arc Lengt Paramterization}
	\subsection{Egzact Diagonalization (ED)}

\section{Polarization}
	\subsection{Classical Polarization Calculation}
	\subsection{Polarization using ED}

\section{Open Questions}
	
\appendix
\section{Mathematics}
	In this appendix I will write down some useful mathematical expressions, proofs or identities that are used in the discussion above
		\subsection{Gaussian Integrals}
			% Guassian integrals

\begin{problem}[\label{PureQuadraticGaussInt}Purely quadratic case]
\begin{equation}
I_{\text{pure gauss.}}(a) = \int_{-\infty}^{\infty} {\rm{d}}x\, e^{ax^2}
\end{equation} 
A way to solve this integral is to intorduce $I(a)^2$, go to a polar coordinate representation by a simple substitution, then integrate over the polar angle $\phi$ and finally take the square root of the result.
\begin{align}
	I(a)^2 &= \int_{-\infty}^{\infty} {\rm{d}}x_1 \int_{-\infty}^{\infty} {\rm{d}} x_2 e^{ax_1^2} e^{ax_2^2} \\
	&= \iint_{-\infty}^{\infty} {\rm{d}}x_1 {\rm{d}}x_2 \, e^{a(x_1^2 + x_2^2)}
\end{align}
Now one can go from ${\rm{d}}x_1 {\rm{d}}x_2$ to ${\rm{d}}r {\rm{d}}\phi$, by the $x_1 = r \cos(\phi)$ and $x_2 = r\sin(\phi)$. Meaning that $x_1^2 + x_2^2 = r^2$ replaces the exponent. Calculating the Jacobian for the change of the metric
\begin{equation}
	J = \left| 
	\begin{matrix}
	\frac{\partial x_1}{\partial r} & \frac{\partial x_1}{\partial \phi} \\
	\frac{\partial x_2}{\partial r} & \frac{\partial x_2}{\partial \phi} 
	\end{matrix}
	\right| = \left|
	\begin{matrix}
	\cos(\phi) & -r\sin(\phi)\\
	\sin(\phi) & r\cos(\phi)\\
	\end{matrix}
	\right| = r
\end{equation}

\begin{align}
	I(a)^2 &= \int_{0}^{\infty} {\rm{d}}r \int_{0}^{2\pi}{\rm{d}}\phi \, re^{a(r^2)} \\
	&= 2\pi \int_{0}^{\infty} {\rm{d}}r \, re^{a(r^2)}
\end{align}
It is convinient to make another substitution $q = -r^2 \rightarrow {\rm{d}}q = -2r {\rm{d}}r$.
\begin{align}
	I(a)^2 &= -2\pi \int_{0}^{\infty} {\rm{d}}q \, \frac{1}{2} e^{-a\,q} \\
	&= \frac{\pi}{-a} \\
	\Aboxed{\int_{-\infty}^{\infty} {\rm{d}}x\, e^{ax^2} &= \sqrt{\frac{\pi}{-a}}}\\
	 &\text{ assuming that $\Real{a} \leq 0$} \nonumber
\end{align}
\end{problem}

\begin{problem}[\label{GeneralQuadraticGaussInt}General Gaussian Integral]
\begin{equation}
	I_{\text{Gen.gauss.}}(a, b, c) = \int_{-\infty}^{\infty} {\rm{d}}x\, e^{ax^2 + bx + c}
\end{equation}
The second order polinomial in the exponent can be written as a square: $ \alpha(x + \beta)^2 - \gamma = ax^2 + bx + c$. Identifying the coefficients on the l.h.s. we find that $\alpha = a;\, \beta = \frac{b}{2a};$ and $\gamma = c - \frac{b^2}{4a^2}$. Now rewrite the integral using the squared form of the polynomial
\begin{equation}
	I(a, b, c) = \int {\rm{d}}x\, e^{a(x + \frac{b}{2a})^2 + c - \frac{b^2}{4a^2}}
\end{equation}
where the last two terms in the exponent are just constant factors which can be placed outside of the integral, leaving the $a(x + \frac{b}{2a})^2$ term. Substitute in $y = x + \frac{b}{2a}$ (${\rm{d}}y = {\rm{d}}x$)
\begin{equation}
	I(a, b, c) = e^{c - \frac{b^2}{4a^2}} \int_{-\infty}^{\infty} {\rm{d}}y\, e^{ay^2}.
\end{equation}
Integral on the l.h.s. is the same as in problem \ref{PureQuadraticGaussInt}, giving the answer as
\begin{equation}
	\boxed{\int_{-\infty}^{\infty} {\rm{d}}x\, e^{ax^2 + bx + c} = e^{c - \frac{b^2}{4a^2}} \sqrt{\frac{\pi}{-a}}}
\end{equation}
\end{problem}







		\subsection{Cauchy's Resiude Theorem}\label{CRT}
		\subsection{Besel Problem}
			\begin{equation}
\sum_{n = 1}^\infty \frac{1}{n^2} = \frac{\pi^2}{6}
\end{equation}


\paragraph{Fourier Way}

Let's consider first an arbitrary function $f(x)$. One could wirte the fourier series of this function as
\begin{equation}
f(x) = \frac{a_0}{2} + \sum_{n=1}^\infty \left\lbrace a_n \cos\left( \frac{n \pi l}{L}  \right) + b_n \sin \left(  \frac{n \pi l}{L} \right)  \right\rbrace,
\end{equation}
where $l\in [l_i,l_f]$ and $L = l_f - l_i$. Let $f(x) = x^2$ and $l_i = -l_f = -\pi$ so that $L = 2\pi$. The coefficients can be calculated as
\begin{align}
a_0 &= \frac{2}{L} \int_{-\pi}^\pi f(x) dx = \frac{2}{\pi} \int_0^\pi dx\, x^2 = \frac{2}{3}\pi^2
\end{align}

\begin{align}
	a_n &= \frac{2}{L} \int_{-\pi}^\pi dx\, f(x) \cos\left(\frac{n \pi 2 x}{L} \right) \\ \nonumber
		&= \frac{2}{\pi }\left.\left(  \frac{x^2 \sin(nx)}{n} \right)\right|_0^\pi - \frac{2}{\pi} \int_0^\pi dx\, \frac{2x 		\sin(nx)}{n}\\ \nonumber
		&= \frac{4 (-1)^n}{n^2}.
\end{align}
Considering the problem at hand, consider the function as $x^2$, so $b_n = 0$ because $x^2$ is even. The Fourier series then takes the form
\begin{align}
x^2 &= \frac{2\pi^2}{3\cdot 2} + \sum_{n=1}^\infty \frac{4 (-1)^n}{n^2} \cos\left( \frac{2n\pi x}{L}  \right) \\
&= \frac{\pi^2}{3} + \sum_{n=1}^\infty \frac{4 (-1)^n}{n^2} (-1)^n. \nonumber
\end{align} Choose $x^2$ to be equal to 0,
\begin{equation}
\pi^2 = \frac{\pi^2}{3} + \sum_{n=1}^\infty \frac{4}{n^2}.
\end{equation} Then subtract the $a_0$ term and devide by 4

\begin{equation}
\frac{\pi^2}{6} = \sum_{n=1}^\infty \frac{1}{n^2}
\end{equation}

\paragraph{Complex Integral Way}

Consider the following integral:
\begin{equation}
I = \int_C \frac{1}{z^2} f(z) dz.
\end{equation} Let $f(z) $ be the Fermi function:
\begin{equation}
f(z) = \frac{1}{1+ e^{i\pi z}},
\end{equation} then we have several first order poles at $z=\pm (2n + 1 ) =p_n$ and a second order pole at $z=0$.

The residue from the 1st order poles:
\begin{equation}
Res(f(z),p_n) = \lim_{z\rightarrow p_n} (z-p_n) f(z) = \frac{
i}{\pi p_n^2},
\end{equation} and from the 2nd order pole:
\begin{equation}
Res(f(z),0) = \lim_{z\rightarrow 0} \frac{d}{dz} (z^2 f(z)) = \frac{-\pi i}{4}.
\end{equation}

Writing now Cauchy's residue theorem \ref{CRT}
\begin{equation}
\int_C g(z) dz = 2\pi i \sum Res(g(z))
\end{equation}

\begin{equation}
\int_C f(z) dz = 2 \pi i \left(  \frac{
-\pi i}{4} \right) + 2 \pi i \sum_{n=0}^\infty \frac{2i}{\pi (2n+1)^2} = 0
\end{equation}
\begin{align}
\frac{\pi i }{4} &= \sum_{n=0}^\infty \frac{2i}{\pi (2n+1)^2} \\
\frac{\pi^2}{8} &= \sum_{n=0}^\infty \frac{1}{(2n+1)^2}\\
&= \frac{3}{4}\sum_{n=1}^\infty \frac{1}{n^2}
\end{align}

\begin{equation}
\frac{\pi^2}{6} = \sum_{n=1}^\infty \frac{1}{n^2}
\end{equation}
		
\section{Algorithms and Methods}
		\subsection{Monte Carlo and Simulated Annealing}
		\subsection{Minimalization by Nelder-Mead algorithm}

\section{Links, Data and other stuff}	



\cite{aa}, \cite{bb}

\newpage
\begin{thebibliography}{3}

\bibitem{aa} Anna Anderegg, 2022, Example 1.

\bibitem{bb} Brenda Bradshaw, 2021, Example 2.

\end{thebibliography}

\end{document}