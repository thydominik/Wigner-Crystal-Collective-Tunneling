% LatexTemplate-------------------------------------------------------------------------
% Rules: 	- Remember keep it light!
%			- Comment out packages that are not in use
%			- Only put in packages with clear intentions that are stated with a comment
% --------------------------------------------------------------------------------------
% Documentclass and packages
	%\documentclass[aps, twocolumns, showpacs, preprintnumbers, superscriptaddress]{revtex4-2}
	\documentclass[reprint, twocolumn, superscriptaddress, preprintnumbers, aps]{revtex4-2}
	
	\usepackage[english]{babel} % This package manages culturally-determined typographical (and other) rules
	\usepackage{amsmath}		% Lot of math input
	\usepackage{amsthm}
	\usepackage{amsfonts}	 	% Mathematical fonts like \mathbb{R}...
	\usepackage{physics}		% for hbar and other  physics related notations
	\usepackage[usenames,dvipsnames]{xcolor}			% use if there are colors in the text
	\usepackage{graphicx}		% for figures

% 
	\usepackage{hyperref} % Makes references, links, toc clickable with changable color
	\hypersetup{
    	colorlinks=true,
    	citecolor=PineGreen,
    	filecolor=Peach,
    	linkcolor=MidnightBlue,
    	urlcolor=purple
	}
%
	\usepackage{import}

% Visual elements
\numberwithin{equation}{section}		% Makes the numbering of the equations numbered by the section/appendix (###) -> (#.##)

% Latex template:
	% Update the original 'Commands.tex' file then copy it to the project that you work on, to keep an updated and uniform latex sintax, ty!

% MATHEMATICS
	% Vectors & Matricies
		\newcommand{\vect}[1]{\,\underline{#1}}		% Creates an underlined vector, good for 3D ones
		\newcommand{\bv}[1]{\,{\bold{#1}}}				% Creates a bold vector
		\newcommand{\mtx}[1]{\,\underline{\underline{#1}}}		% Creates a doubly underlined letter, as matrix
		\newcommand{\bmtx}[1]{\,\underline{\underline{#1}}}		% Creates a bold letter as matrix
		
		\newcommand{\absv}[1]{\,\lVert #1 \rVert}
		\newcommand{\diad}[2]{\, {\bold{#1} \circ \bold{#2}}}
	
	% 
		\newcommand{\R}{\mathbb{R}} 	% set for real numbers
		\newcommand{\Q}{\mathbb{Q}}		% set for cplx numbers
		\newcommand{\argu}[1]{\!\left({#1} \right)} 	% identity
		
		\newcommand{\Real}[1]{\mathbb{Re} \left( #1 \right)}
% PHYSICS


% Latex Environments:
	% Update the original 'Environments.tex' file then copy it to the project that you work on, to keep an updated and uniform latex sintax, ty!

% Notes elements:
	% Theorem
		\newtheorem{theorem}{Theorem}[section]		%\begin{} and \end{} 'theorem', with printed 'Theorem ###' where the numbering will depend on the section it is placed in
	% Corollary
		\newtheorem{corollary}{Corollary}[theorem]
	% Definition
		\newtheorem{definition}{Definition}[section]
	% Lemma
		\newtheorem{lemma}[theorem]{Lemma}
	% Remark
		\theoremstyle{remark}
		\newtheorem*{remark}{Remark}
		
	\theoremstyle{definition}
		\newtheorem{problem}{Problem}[section]
		\newtheorem{example}{Example}[section] %\begin{} and \end{} 'example', with printed 'Example ###' where the numbering will depend on the section it is placed in
		\newtheorem{exercise}{Exercise}[section] 	% Initially for Bevezető kalkulus notes
		\newtheorem{recexercise}{recommended Exercise}[section]		% Initially for Bevezető kalkulus notes

\usepackage{blindtext}

\begin{document}

\title{Wigner Crystal Collective tunneling - A theoretical investigation}
\author{Szombathy Dominik}
	\email{szombathy.dominik@gmail.com \\ szombathy.dominik@edu.bme.hu}
	\homepage{www.internet.hu/secrets}
	\affiliation{Budapest University of Technology}
	
	
\date{\today}



\begin{abstract}
	% Generic abstract

The documentation of the title's topic, Wigner crystal collective tunneling. Motivated by experiments, where in an effectively one dimensional system a wigner crystal for up to 7 electrons were observed. This work aims to investigate the phenomena from a theoreitcal point of view using ED, Instanton and DMRG techniques.

	\begin{description}
		\item[Usage]
This note aims to document every major developement in the project, explain the necessary 'new' physics (at least new for me), derive or proof mathematical concepts that the investigation is dependent upon, and be a somewhat self-contained material for me and possibly others during the research process. 
		\item[Structure]
The structure is as follows: Starting from elementary physics concepts that describes the environment that we are trying to understand, then developing the path integral tools along Feynman's book and Milnikov's work, these are followed by Monte-Carlo simulations, ED calculations and Polarization calculations,then ends with open quaestions, hopefully with some answers. The notes ends with the Appendix containing the necessary math, programming and other important materials.
	\end{description}
\end{abstract}

\maketitle

\tableofcontents


\section{Wigner Crystal}\blindtext
	\nt{this is a note, neat isn't it?}
\section{Path Integral Formalism} \blindtext
	\subsection{Free Particle}\blindtext
	\subsection{Harmonic Oscillator}\blindtext
	\subsection{One Particle Tunneling}\blindtext
		\subsubsection{One Particle Hamiltonian}\blindtext
		\subsubsection{Action Integral Calculation}\blindtext
		\subsubsection{Tunneling calculation in quartic potentials}\blindtext
\section{Experimental Considerations}\blindtext
	\subsection{Modeling the Potential and Fitting}\blindtext
	\subsection{Polarization data}\blindtext
	\subsection{Scaling}\blindtext
\section{Many Body Tunneling Calculations - Instanton}\blindtext
	\subsection{Quartic Hamiltonian model}\blindtext
		\subsubsection{Rescaling and Dimensionless Hamiltonian}	\blindtext
	\subsection{Milnikov method}\blindtext
	\subsection{Classical Equilibrium Positions}\blindtext
	\subsection{Harmonic Oscillator Eigenfrequencies adn Eigenmodes}\blindtext
	\subsection{Trajectory Calculation}\blindtext
	\subsection{Arc Lengt Paramterization}\blindtext
	\subsection{Egzact Diagonalization (ED)}\blindtext
\section{Polarization}\blindtext
	\subsection{Classical Polarization Calculation}\blindtext
	\subsection{Polarization using ED}\blindtext
\section{Open Questions}\blindtext


\appendix
\section{Mathematics}
		\subsection{Gaussian Integrals}
			\import{./Mathematics/}{Gaussian_Integrals.tex}
		\subsection{Cauchy's Resiude Theorem}\label{CRT}
		\subsection{Besel Problem}
			\import{./Mathematics/}{Besel_Problem.tex}
		
\section{Algorithms and Methods}
		\subsection{Monte Carlo and Simulated Annealing}
		\subsection{Minimalization by Nelder-Mead algorithm}

\section{Links, Data and other stuff}	



\cite{aa}, \cite{bb}

\newpage
\begin{thebibliography}{3}

\bibitem{aa} Anna Anderegg, 2022, Example 1.

\bibitem{bb} Brenda Bradshaw, 2021, Example 2.

\end{thebibliography}
\end{document}