% LatexTemplate-------------------------------------------------------------------------
% Rules: 	- Remember keep it light!
%			- Comment out packages that are not in use
%			- Only put in packages with clear intentions that are stated with a comment
% --------------------------------------------------------------------------------------
% Documentclass and packages
	%\documentclass[aps, twocolumns, showpacs, preprintnumbers, superscriptaddress]{revtex4-2}
	\documentclass[reprint, twocolumn, superscriptaddress, preprintnumbers, aps]{revtex4-2}
	
	\usepackage[english]{babel} % This package manages culturally-determined typographical (and other) rules
	\usepackage{amsmath}		% Lot of math input
	\usepackage{amsthm}
	\usepackage{amsfonts}	 	% Mathematical fonts like \mathbb{R}...
	\usepackage{physics}		% for hbar and other  physics related notations
	\usepackage[usenames,dvipsnames]{xcolor}			% use if there are colors in the text
	\usepackage{graphicx}		% for figures

% 
	\usepackage{hyperref} % Makes references, links, toc clickable with changable color
	\hypersetup{
    	colorlinks=true,
    	citecolor=PineGreen,
    	filecolor=Peach,
    	linkcolor=MidnightBlue,
    	urlcolor=purple
	}
%
	\usepackage{import}

% Visual elements
\numberwithin{equation}{section}		% Makes the numbering of the equations numbered by the section/appendix (###) -> (#.##)

% Title, Author and other stuff
\title{Theoretical modeling of the collective tunneling of a Wigner necklace\\}
\author{Dominik szombathy}
\date{\today}


\begin{document}
\maketitle

%\pagestyle{empty}
\tableofcontents

%\setcounter{page}{1}
\pagestyle{plain}

\newpage
\section{Path Integral Formalism (introduction)}

\newpage
\section{Wigner Crystals}

\newpage
\section{Instanton Formalism}
	\subsection{Free Particle}
	\subsection{Harmonic Oscillator}
	\subsection{One particle tunneling}
		\subsubsection{One Particle Hamiltonian}
		\subsubsection{Action Integral Calculation}
		\subsubsection{Tunneling calculation in quartic potentials}

\newpage
\section{Experimental Considerations}
	\subsection{Modeling the Potential and Fitting}
	\subsection{Polarization data}
	\subsection{Scaling}
		
\newpage
\section{Many Body Tunneling Calculations - Instanton}
	\subsection{Quartic Hamiltonian model}
		\subsubsection{Rescaling and Dimensionless Hamiltonian}	
	\subsection{Milnikov method}
	\subsection{Classical Equilibrium Positions}
	\subsection{Harmonic Oscillator Eigenfrequencies adn Eigenmodes}
	\subsection{Trajectory Calculation}
	\subsection{Arc Lengt Paramterization}
	\subsection{Egzact Diagonalization (ED)}

\newpage
\section{Polarization}
	\subsection{Classical Polarization Calculation}
	\subsection{Polarization using ED}
	

\newpage	
\appendix
\section{Mathematics}
	In this appendix I will write down some useful mathematical expressions, proofs or identities that are used in the discussion above
		\subsection{Gaussian Integrals}
		\subsection{Besel Problem}
			\input{Besel_Problem.tex}

\newpage		
\section{Algorithms and Methods}
		\subsection{Monte Carlo and Simulated Annealing}
		\subsection{Minimalization by Nelder-Mead algorithm}
	
\end{document}