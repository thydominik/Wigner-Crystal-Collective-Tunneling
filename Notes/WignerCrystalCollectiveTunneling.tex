% LatexTemplate-------------------------------------------------------------------------
% Rules: 	- Remember keep it light!
%			- Comment out packages that are not in use
%			- Only put in packages with clear intentions that are stated with a comment
% --------------------------------------------------------------------------------------
% Documentclass and packages
	%\documentclass[aps, twocolumns, showpacs, preprintnumbers, superscriptaddress]{revtex4-2}
	\documentclass[reprint, twocolumn, superscriptaddress, preprintnumbers, aps]{revtex4-2}
	% LANGUAGE:
		\usepackage[english]{babel} % This package manages culturally-determined typographical (and other) rules
	% MATH & PHYSICS NOTATION
		\usepackage{amsmath}		% Lot of math input
		\usepackage{amsthm, mathtools}
		\usepackage{amsfonts}	 	% Mathematical fonts like \mathbb{R}...
		\usepackage{physics}		% for hbar and other  physics related notations
	% LOOKS & GEOMETRY & COLORS
		\usepackage[tmargin = 2.3cm, rmargin = 4cm, lmargin = 3cm, margin = 3cm, bmargin = 3cm, footskip = 0.5cm]{geometry}
		\usepackage[dvipsnames]{xcolor}			% use if there are colors in the text
		\usepackage[most,many,breakable]{tcolorbox}
		\usepackage{graphicx}		% for figures
		\usepackage{enumitem}
		\usepackage{hyperref} % Makes references, links, toc clickable with changable color
		\hypersetup{
    		colorlinks=true,
    		citecolor=PineGreen,
    		filecolor=Peach,
    		linkcolor=MidnightBlue,
    		urlcolor=purple
			}
	% MISC
		\usepackage{import}
		%\usepackage{lipsum}

% Visual elements
\numberwithin{equation}{section}		% Makes the numbering of the equations numbered by the section/appendix (###) -> (#.##)

% Title, Author and other stuff
\title{Theoretical modeling of the collective tunneling of a Wigner necklace\\}
\author{Dominik szombathy}
\date{\today}


\begin{document}
\maketitle

%\pagestyle{empty}
\tableofcontents

%\setcounter{page}{1}
\pagestyle{plain}

\newpage
\section{Path Integral Formalism (introduction)}

\newpage
\section{Wigner Crystals}

\newpage
\section{Instanton Formalism}
	\subsection{Free Particle}
	\subsection{Harmonic Oscillator}
	\subsection{One particle tunneling}
		\subsubsection{One Particle Hamiltonian}
		\subsubsection{Action Integral Calculation}
		\subsubsection{Tunneling calculation in quartic potentials}

\newpage
\section{Experimental Considerations}
	\subsection{Modeling the Potential and Fitting}
	\subsection{Polarization data}
	\subsection{Scaling}
		
\newpage
\section{Many Body Tunneling Calculations - Instanton}
	\subsection{Quartic Hamiltonian model}
		\subsubsection{Rescaling and Dimensionless Hamiltonian}	
	\subsection{Milnikov method}
	\subsection{Classical Equilibrium Positions}
	\subsection{Harmonic Oscillator Eigenfrequencies adn Eigenmodes}
	\subsection{Trajectory Calculation}
	\subsection{Arc Lengt Paramterization}
	\subsection{Egzact Diagonalization (ED)}

\newpage
\section{Polarization}
	\subsection{Classical Polarization Calculation}
	\subsection{Polarization using ED}
	

\newpage	
\appendix
\section{Mathematics}
	In this appendix I will write down some useful mathematical expressions, proofs or identities that are used in the discussion above
		\subsection{Gaussian Integrals}
		\subsection{Besel Problem}
			\begin{equation}
\sum_{n = 1}^\infty \frac{1}{n^2} = \frac{\pi^2}{6}
\end{equation}


\paragraph{Fourier Way}

Let's consider first an arbitrary function $f(x)$. One could wirte the fourier series of this function as
\begin{equation}
f(x) = \frac{a_0}{2} + \sum_{n=1}^\infty \left\lbrace a_n \cos\left( \frac{n \pi l}{L}  \right) + b_n \sin \left(  \frac{n \pi l}{L} \right)  \right\rbrace,
\end{equation}
where $l\in [l_i,l_f]$ and $L = l_f - l_i$. Let $f(x) = x^2$ and $l_i = -l_f = -\pi$ so that $L = 2\pi$. The coefficients can be calculated as
\begin{align}
a_0 &= \frac{2}{L} \int_{-\pi}^\pi f(x) dx = \frac{2}{\pi} \int_0^\pi dx\, x^2 = \frac{2}{3}\pi^2
\end{align}

\begin{align}
	a_n &= \frac{2}{L} \int_{-\pi}^\pi dx\, f(x) \cos\left(\frac{n \pi 2 x}{L} \right) \\ \nonumber
		&= \frac{2}{\pi }\left.\left(  \frac{x^2 \sin(nx)}{n} \right)\right|_0^\pi - \frac{2}{\pi} \int_0^\pi dx\, \frac{2x 		\sin(nx)}{n}\\ \nonumber
		&= \frac{4 (-1)^n}{n^2}.
\end{align}
Considering the problem at hand, consider the function as $x^2$, so $b_n = 0$ because $x^2$ is even. The Fourier series then takes the form
\begin{align}
x^2 &= \frac{2\pi^2}{3\cdot 2} + \sum_{n=1}^\infty \frac{4 (-1)^n}{n^2} \cos\left( \frac{2n\pi x}{L}  \right) \\
&= \frac{\pi^2}{3} + \sum_{n=1}^\infty \frac{4 (-1)^n}{n^2} (-1)^n. \nonumber
\end{align} Choose $x^2$ to be equal to 0,
\begin{equation}
\pi^2 = \frac{\pi^2}{3} + \sum_{n=1}^\infty \frac{4}{n^2}.
\end{equation} Then subtract the $a_0$ term and devide by 4

\begin{equation}
\frac{\pi^2}{6} = \sum_{n=1}^\infty \frac{1}{n^2}
\end{equation}

\paragraph{Complex Integral Way}

Consider the following integral:
\begin{equation}
I = \int_C \frac{1}{z^2} f(z) dz.
\end{equation} Let $f(z) $ be the Fermi function:
\begin{equation}
f(z) = \frac{1}{1+ e^{i\pi z}},
\end{equation} then we have several first order poles at $z=\pm (2n + 1 ) =p_n$ and a second order pole at $z=0$.

The residue from the 1st order poles:
\begin{equation}
Res(f(z),p_n) = \lim_{z\rightarrow p_n} (z-p_n) f(z) = \frac{
i}{\pi p_n^2},
\end{equation} and from the 2nd order pole:
\begin{equation}
Res(f(z),0) = \lim_{z\rightarrow 0} \frac{d}{dz} (z^2 f(z)) = \frac{-\pi i}{4}.
\end{equation}

Writing now Cauchy's residue theorem \ref{CRT}
\begin{equation}
\int_C g(z) dz = 2\pi i \sum Res(g(z))
\end{equation}

\begin{equation}
\int_C f(z) dz = 2 \pi i \left(  \frac{
-\pi i}{4} \right) + 2 \pi i \sum_{n=0}^\infty \frac{2i}{\pi (2n+1)^2} = 0
\end{equation}
\begin{align}
\frac{\pi i }{4} &= \sum_{n=0}^\infty \frac{2i}{\pi (2n+1)^2} \\
\frac{\pi^2}{8} &= \sum_{n=0}^\infty \frac{1}{(2n+1)^2}\\
&= \frac{3}{4}\sum_{n=1}^\infty \frac{1}{n^2}
\end{align}

\begin{equation}
\frac{\pi^2}{6} = \sum_{n=1}^\infty \frac{1}{n^2}
\end{equation}

\newpage		
\section{Algorithms and Methods}
		\subsection{Monte Carlo and Simulated Annealing}
		\subsection{Minimalization by Nelder-Mead algorithm}
	
\end{document}